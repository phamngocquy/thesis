\documentclass[12pt]{report}
\usepackage[fontsize=13pt]{scrextend}
\usepackage[utf8]{vietnam}
\usepackage[utf8]{inputenc}
\usepackage[vietnamese]{babel}

\title{sis}
\author{phamngocquy97 }
\date{February 2019}

\usepackage{natbib}
\usepackage{graphicx}

\begin{document}
	
%-----MAIN-----%
\newpage
\pagenumbering{arabic}
\setcounter{page}{1}
\chapter{Phương pháp kiểm tra sự tuân thủ mẫu thiết kế cho dự án sử dụng Java}
Mẫu thiết kế là tập hợp các luật nhằm mô tả cách giải quyết một vấn đề trong thiết kế. Với những dự án Java nói riêng. Ở các mẫu thiết kế hướng đối tượng, thường thể hiện mối quan hệ giữa các lớp, các đối tượng với nhau. Do đó, để kiểm tra sự tuân thủ mẫu thiết kế trong mã nguồn dự án.
Phương pháp kiểm tra sự tuân thủ mẫu thiết kế được mô tả như Hình 3.1. Để kiểm tra sự tuân thủ mẫu thiết kế trong mã nguồn. Đầu tiên, dữ liệu đầu vào được tiền xử lý thành cây cú pháp, thông qua cây cú pháp tiến hành phân tích phụ thuộc bên trong mã nguồn, xây dựng đồ thị phụ thuộc. Phân tích đồ thị phụ thuộc của mã nguồn mà đồ thị phụ thuộc của mẫu thiết kế đầu vào nhằm kiểm tra sự tuân thủ mẫu thiết kế trong mã nguồn đầu vào
\section{Tiền xử lý mã nguồn Java}

\newpage
\section{Phân tích cấu trúc mã nguồn}
\subsection{Phân tích phụ thuộc giữa các thành phần trong mã nguồn}
\subsection{Xây dựng đồ thị phụ thuộc từ cây cấu trúc}
\subsection{Ví dụ minh họa}

\newpage
\section{Kiểm tra sự tuân thủ mẫu thiết kế bên trong mã nguồn}

\bibliographystyle{plain}
\bibliography{references}
\end{document}
